\usepackage{verbatim}
\usepackage[thmmarks]{ntheorem}
%Definiciones
\def\ser{\leavevmode\raise.585ex\hbox{\small er}} %1er
\def\sera{\leavevmode\raise.585ex\hbox{\small era}} %1era
%Similarmente para las demas

\def\sen{\mathop{\mbox{\normalfont sen}}\nolimits} %Definicion de sen para evitar utilizar \sin

\def\max{\mathop{\mbox{\normalfont m\’ax}}\limits} %Definicion de max que admite subindice
\def\sup{\mathop{\mbox{\normalfont sup}}\limits}
\def\inf{\mathop{\mbox{\normalfont inf}}\limits}

%\renewcommand\qedsymbol{$\blacksquare$} %Cambiando el QED por un cuadro negro

\newcommand{\introduction}{\chapter*{Introducci\'on}}

\begin{comment}
%\theoremstyle{definition}
  \newtheorem{definition}{Definici\'on}[chapter]
  \newtheorem{example}{Ejemplo}[chapter]

%\theoremstyle{plain}
  \newtheorem{theorem}{Teorema}[chapter]
  \newtheorem{corollary}{Corolario}[theorem]
  \newtheorem{proposition}{Proposici\'on}[chapter]

%\theoremstyle{remark}
  \newtheorem{remark}{Observaci\'on}[chapter]
\end{comment}


\theoremstyle{plain}
\theorembodyfont{\upshape}
\theoremseparator{.}
\newtheorem{definition}{Definici\'on}[section]

\theoremstyle{plain}
\theorembodyfont{\upshape}
\theoremseparator{.}
\newtheorem{observations}{Observaciones}[section]

\theoremstyle{plain}
\theorembodyfont{\upshape}
\theoremseparator{.}
\newtheorem{observation}{Observaci\'on}[section]

\theoremstyle{plain}
\theorembodyfont{\upshape}
\theoremseparator{.}
\newtheorem{proposition}{Proposici\'on}[section]


\theoremstyle{plain}
\theorembodyfont{\upshape}
\theoremseparator{}
\newtheorem{example}{Ejemplo}[section]

\theoremstyle{plain}
\theorembodyfont{\upshape}
\theoremseparator{}
\newtheorem{exercices}{Ejercios}[section]


\theoremstyle{plain}
\theoremheaderfont{\normalfont\bfseries}\theorembodyfont{\itshape}
\theoremseparator{.}
\newtheorem{theorem}{Teorema}[section]

\theoremstyle{plain}
\theoremheaderfont{\normalfont\bfseries}\theorembodyfont{\itshape}
\theoremseparator{.}
\newtheorem{corollary}{Corolario}[section]

\theoremstyle{plain}
\theoremindent0.5cm
%\theoremnumbering{greek}
\newtheorem{lemma}{Lema}[section]

%Inicio de la definicion del ambiente Proof
\theoremheaderfont{\sc}\theorembodyfont{\upshape}
\theoremstyle{nonumberplain}
\theoremseparator{:}
\theoremsymbol{%\rule{1ex}{1ex}
\qedsymbol%
}
\newtheorem{Proof}{Prueba}

\theoremheaderfont{\sc}\theorembodyfont{\upshape}
\theoremstyle{nonumberplain}
\theoremseparator{:}
\theoremsymbol{%\rule{1ex}{1ex}
\qedsymbol%
}
\newtheorem{solution}{Solución}
%Fin
