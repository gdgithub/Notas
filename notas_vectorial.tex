\documentclass[letterpaper, 12pt]{article} %estableciendo la clase de documento como exam con papel legal y sizefont 12
\usepackage{amsmath, amssymb, amsfonts, latexsym}
\usepackage[utf8]{inputenc}
\usepackage[spanish]{babel}
\input{MyMacros.tex}

\begin{document}
\title{Algunos conceptos del calculo vectorial}
\author{Ivan Gil}
\maketitle

\section{Derivada}

Iniciaremos la motivación con la noción de derivada en una variable real. Luego extenderemos estas ideas
al espacio Euclideo $\mathbb{R}^{n}$:

\begin{definition}[Derivada en $\mathbb{R}$]
Sea $f: I \subset \mathbb{R} \longrightarrow \mathbb{R}$ y $x_{0} \in I$. Se dice que que $f$ es
{\it diferenciable} en $x_{0}$ si existe el limite siguiente

$$\lim_{x \longrightarrow x_{0}} \frac{f(x) - f(x_{0})}{x-x_{0}} \mbox{.}$$

Cuando el limite anterior existe, a su valor lo llamamos {\it derivada} de $f$ en $x_{0}$ y lo representamos
por $f'(x_{0})$. Ademas, a la función que asigna a cada $x \in I$ la derivada de $f$ en $x$ la llamamos
{\it funcion derivada} de $f$ y representamos por $f'(x)$.
\end{definition}

\begin{observation}
Otra manera de expresar la derivada de $f$ en $x_{0}$ es

$$\lim_{x \longrightarrow x_{0}} \frac{f(x) - f(x_{0}) - f'(x_{0})(x-x_{0})}{x-x_{0}} = 0$$

Mas aun,

\[
\lim_{x \longrightarrow x_{0}} \left \vert \frac{f(x) - f(x_{0}) - f'(x_{0})(x-x_{0})}{x-x_{0}}\right \vert =
\lim_{x \longrightarrow x_{0}} \frac{\vert f(x) - f(x_{0}) - f'(x_{0})(x-x_{0}) \vert}{\vert x-x_{0} \vert} = 0
\]

Este ultimo resultado es util para generalizar la noción de derivada a funciones
$f : \mathbb{R}^{m} \longrightarrow \mathbb{R}^{n}$
\end{observation}

\begin{definition}[Derivada]
Sea $f : U \subset \mathbb{R}^{m} \longrightarrow \mathbb{R}^{n}$ y $x_{0} \in U$. Se dice que que $f$ es
{\it diferenciable} en $x_{0}$ si existe una matriz $M_{f}(x_{0}) \in \mathcal{M}_{n \times m}$ tal que

$$\lim_{x \longrightarrow x_{0}} \frac{|| f(x) - f(x_{0}) - M_{f}(x_{0})(x-x_{0}) ||}{|| x-x_{0} ||} = 0 \mbox{.}$$

En tal caso se dice que $M_{f}(x_{0})$ es la derivada de $f$ en $x_{0}$. Ademas, a la función que asigna
a cada $x \in U$ la derivada de $f$ en $x$ la llamamos función derivada de $f$ y representamos por $M_{f}(x)$.
\end{definition}

\begin{observation}
Esto ultimo nos muestra que, en general, la derivada de una función $f : \mathbb{R}^{m} \longrightarrow \mathbb{R}^{n}$
 es una matrix de orden $n \times m$.
\end{observation}

\begin{definition}[Matriz Jacobiana]
Sea $f : U \subset \mathbb{R}^{m} \longrightarrow \mathbb{R}^{n}$ tal que $f \in \mathcal{C}^{1}(U)$ y $x_{0} \in U$.
Llamamos {\it matriz Jacobiana}, y representamos como $J_{f}(x)$ a la matriz definida por

$$J_{f}(x) = \sum_{i=1}^{n}\sum_{j=1}^{m} \frac{\partial f \cdot e_{i}}{\partial x_{j}} \delta_{ij}$$
\end{definition}

\begin{observation}
Si $f : U \subset \mathbb{R}^{m} \longrightarrow \mathbb{R}^{n}$, entonces existen
$f_{i} : U \subset \mathbb{R}^{m} \longrightarrow \mathbb{R}$ tales que $f = (f_{1}, f_{2}, \cdots, f_{n})$.\\

Como $f \cdot e_{i} = f_{i}$, podemos escribir la matriz Jacobiana de $f$ como:

$$J_{f}(x) = \sum_{i=1}^{n}\sum_{j=1}^{m} \frac{\partial f_{i}}{\partial x_{j}} \delta_{ij}$$
\end{observation}

\begin{observation}
La matriz Jacobiana de $f$ en $x_{0}$ es precisamente la matriz $M_{f}(x_{0})$ de la definición de derivada.
Asi que la derivada de una función $f : U \subset \mathbb{R}^{m} \longrightarrow \mathbb{R}^{n}$ en $x_{0} \in U$
es $J_{f}(x_{0})$.
\end{observation}

\newpage
\begin{example}
Encontrar la derivada de las siguientes funciones

\begin{enumerate}
  \item $f: \mathbb{R}^{3} \longrightarrow \mathbb{R}^{2}$ tal que $f(x,y,z) = 3x^{2}yz\hat{i} + 6xyz\hat{j}$\\

  {\sc Solución:}\\
  Solo basta determinar la matriz Jacobiana de $f$. Asi que

  \begin{align*}
  J_{f}(x) &=
  \begin{bmatrix}
  \frac{\partial \, 3x^{2}yz}{\partial x} & \frac{\partial \, 3x^{2}yz}{\partial y} & \frac{\partial \, 3x^{2}yz}{\partial z} \\
  \frac{\partial \, 6xyz}{\partial x} & \frac{\partial \, 6xyz}{\partial y} & \frac{\partial \, 6xyz}{\partial z}
  \end{bmatrix}\\
   &=
  \begin{bmatrix}
  6xyz & 3x^{2}z & 3x^{2}y\\
  6yz & 6xz & 6xy
  \end{bmatrix}
  \end{align*}

  \item $f: \mathbb{R} \longrightarrow \mathbb{R}^{2}$ tal que $f(x) = 2x^{2}\hat{i} + 6x^{3}\hat{j}$\\

  {\sc Solución:}\\
  Solo basta determinar la matriz Jacobiana de $f$. Asi que

  \begin{align*}
  J_{f}(x) &=
  \begin{bmatrix}
  \frac{\partial \, 2x^{2}}{\partial x}\\
  \frac{\partial \, 6x^{3}}{\partial x}
  \end{bmatrix}\\
   &=
  \begin{bmatrix}
  4x\\
  18x^{2}
  \end{bmatrix}
  \end{align*}


\end{enumerate}
\end{example}

\begin{definition}[Diferencial]
Sea $f : U \subset \mathbb{R}^{m} \longrightarrow \mathbb{R}^{n}$ tal que $f \in \mathcal{C}^{1}(U)$. Se
define el {\it diferencial} de $f$, y representamos por $df$, como

$$df = J_{f}(x)dx$$

donde $dx =
\begin{bmatrix}
dx_{1}\\
dx_{2}\\
\vdots\\
dx_{m}
\end{bmatrix}$
\end{definition}

\newpage
\begin{example}
Encontrar el diferencial de la función $f : \mathbb{R}^{3} \longrightarrow \mathbb{R}^{3}$ definida por
$f(x,y,z) = 3\cos(xyz)\hat{i} + 4\sen(xyz)\hat{j} + x^{2}y^{2}z^{2}\hat{k}$ en $x_{0} = (1,1,\pi)$.\\

{\sc Solución:}\\
En primer lugar vamos a determinar la matriz Jacobiana de $f$. Tenemos:


\begin{align*}
J_{f}(x) &=
\begin{bmatrix}
\frac{\partial \, 3\cos(xyz)}{\partial x} & \frac{\partial \, 3\cos(xyz)}{\partial y} & \frac{\partial \, 3\cos(xyz)}{\partial z}\\
\frac{\partial \, 4\sen(xyz)}{\partial x} & \frac{\partial \, 4\sen(xyz)}{\partial y} & \frac{\partial \, 4\sen(xyz)}{\partial z}\\
\frac{\partial \, x^{2}y^{2}z^{2}}{\partial x} & \frac{\partial \, x^{2}y^{2}z^{2}}{\partial y} & \frac{\partial \, x^{2}y^{2}z^{2}}{\partial z}
\end{bmatrix}\\
 &=
\begin{bmatrix}
-3yz\sen(xyz) & -3xz\sen(xyz) & -3xy\sen(xyz)\\
4yz\cos(xyz) & 4xz\cos(xyz) & 4xy\cos(xyz)\\
2xy^{2}z^{2} & 2x^{2}yz^{2} & 2x^{2}y^{2}z
\end{bmatrix}
\end{align*}

Evaluando la matriz Jacobiana en $x_{0}$, se tiene

\begin{align*}
J_{f}(x_{0}) = J_{f}(1,1,\pi) &=
\begin{bmatrix}
-3\pi\sen(\pi) & -3\pi\sen(\pi) & -3\sen(\pi)\\
4\pi\cos(\pi) & 4\pi\cos(\pi) & 4\cos(\pi)\\
2\pi^{2} & 2\pi^{2} & 2\pi
\end{bmatrix}\\
&=
\begin{bmatrix}
0 & 0 & 0\\
-4\pi & -4\pi & -4\\
2\pi^{2} & 2\pi^{2} & 2\pi
\end{bmatrix}
\end{align*}

Por tanto determinamos el diferencial de $f$ en $x_{0}$, como sigue

\begin{align*}
df(x_{0}) = J_{f}(x_{0})dx &=
\begin{bmatrix}
0 & 0 & 0\\
-4\pi & -4\pi & -4\\
2\pi^{2} & 2\pi^{2} & 2\pi
\end{bmatrix}
\begin{bmatrix}
dx\\
dy\\
dz
\end{bmatrix}\\
&=
\begin{bmatrix}
0\\
-4\pi dx - 4\pi dy -4dz\\
2\pi^{2} dx + 2\pi^{2} dy + 2\pi dz
\end{bmatrix}
\end{align*}

\end{example}


\end{document}
